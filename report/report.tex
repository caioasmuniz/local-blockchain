\documentclass{article}
\usepackage{listings}
\usepackage{color}
\usepackage{hyperref}

\title{Atividade Prática sobre Blockchain - Segurança e Auditoria de Sistemas}
\author{Caio Augusto de Souza Muniz, 2050889}
\date{4 de julho de 2022}


\renewcommand\lstlistingname{Quelltext}
\lstset{
    language=C,
    basicstyle=\small,
    numbers=left,
    numberstyle=\tiny,
    frame=tb,
    tabsize=2,
    columns=fixed,
    showstringspaces=false,
    showtabs=false,
    keepspaces,
    commentstyle=\color{red},
    keywordstyle=\color{blue}
}
\begin{document}
\maketitle

\section{Introdução}
A aplicação relatada neste relatório de atividade assíncrona tem como objetivos:
\begin{itemize}
    \item Criação de uma aplicação de blockchain local
    \item Implementação de um algoritmo de prova por trabalho local onde a dificuldade é definida pelo usuário da aplicação
    \item Armazenamento os blocos validados em um arquivo
    \item Implementação de rotina de verificação de integridade dos blocos.
    \item O conteúdo presente nos blocos são \textit{strings}.
\end{itemize}
\section{Ferramentas}
Para o desenvolvimento da aplicação foram utilizadas as seguintes ferramentas:
\begin{itemize}
    \item A linguagem escolhida foi o \textit{Javascript}, através do compilador \textit{Node.js} em sua versão 16.15;
    \item Foi utilizado como base o repositório \href{https://github.com/Savjee/SavjeeCoin}{\textit{Savjee/SavjeeCoin}}, disponível no \textit{GitHub}.
\end{itemize}
\section{Metodologia}
A interface da aplicação é feita pela linha de comando, esta apresenta um menu com as seguintes opções:
\begin{enumerate}
    \item Ler blockchain de um arquivo;
    \item Minerar blocos pendentes;
    \item Selecionar dificuldade de mineração;
    \item Criar novo bloco;
    \item Verificar integridade da blockchain;
    \item Exibir a blockchain;
    \item Salvar a blockchain em um arquivo;
\end{enumerate}
\section{Análise do Impacto da Dificuldade}


\end{document}
