\documentclass{article}
\usepackage{listings}
\usepackage{color}
\usepackage{hyperref}
\usepackage{graphicx}
\graphicspath{ {./src/} }


\title{Atividade Prática sobre Blockchain - Segurança e Auditoria de Sistemas}
\author{Caio Augusto de Souza Muniz, 2050889}
\date{4 de julho de 2022}


\renewcommand\lstlistingname{Quelltext}
\lstset{
    language=C,
    basicstyle=\small,
    numbers=left,
    numberstyle=\tiny,
    frame=tb,
    tabsize=2,
    columns=fixed,
    showstringspaces=false,
    showtabs=false,
    keepspaces,
    commentstyle=\color{red},
    keywordstyle=\color{blue}
}
\begin{document}
\maketitle

\section{Introdução}
A aplicação relatada neste relatório de atividade assíncrona tem como objetivos:
\begin{itemize}
    \item Criação de uma aplicação de blockchain local
    \item Implementação de um algoritmo de prova por trabalho local onde a dificuldade é definida pelo usuário da aplicação
    \item Armazenamento os blocos validados em um arquivo
    \item Implementação de rotina de verificação de integridade dos blocos.
    \item O conteúdo presente nos blocos são \textit{strings}.
\end{itemize}
\section{Ferramentas}
Para o desenvolvimento da aplicação foram utilizadas as seguintes ferramentas:
\begin{itemize}
    \item A linguagem escolhida foi o \textit{Javascript}, através do compilador \textit{Node.js} em sua versão 16.15;
    \item Foi utilizado como base o repositório \href{https://github.com/Savjee/SavjeeCoin}{\textit{Savjee/SavjeeCoin}}, disponível no \textit{GitHub}.
\end{itemize}
\section{Metodologia}
O código-fonte da aplicação está disponível em: \href{https://github.com/caioasmuniz/local-blockchain}{\textit{caioasmuniz/local-blockchain}}

A interface da aplicação é feita pela linha de comando, esta apresenta um menu com as seguintes opções:
\begin{enumerate}
    \item Ler blockchain de um arquivo;
    \item Minerar blocos pendentes;
    \item Selecionar dificuldade de mineração;
    \item Criar novo bloco;
\item Verificar integridade da \textit{blockchain};
\item Exibir a \textit{blockchain};
\item Salvar a \textit{blockchain} em um arquivo;
\end{enumerate}
\section{Análise do Impacto da Dificuldade}
Para a observação do impacto da dificuldade no tempo de execução do algoritmo, foi utilizado o \textit{script} "\textit{test-difficulty.js}", presente no \href{https://github.com/caioasmuniz/local-blockchain}{repositório da atividade}. O \textit{script} continuamente insere e minera uma quantidade de blocos na \textit{blockchain}, enquanto monitora e loga a duração de cada inserção. Este processo foi realizado para cada uma das dificuldades de 1 a 6, tendo 20 blocos gerados e minerados em cada uma das dificuldades. Após a execução do script, foi gerada a seguinte tabela com média e desvio padrão do tempo de execução:
\begin{table}[h]
  \begin{tabular}{| c | c | c |}
    \hline
    Dificuldade & Tempo Médio de execução (ms) & Desvio Padrão (ms) \\
    \hline
    1           & 0.3                          & 0.4775             \\
    \hline
    2           & 2.95                         & 3.4301             \\
    \hline
    3           & 15.4                         & 15.6277            \\
    \hline
    4           & 287.4                        & 207.9394           \\
    \hline
    5           & 4366.15                      & 4004.0907          \\
    \hline
    6           & 87177.7                      & 59671.2204         \\
    \hline
  \end{tabular}
  \caption{Tabela de tempo médio de execução e desvio padrão de acordo com a dificuldade}
\end{table}

Além disso, foi gerado o seguinte gráfico do tempo médio de mineração pela dificuldade:
\begin{figure}[h]
  \includegraphics[width=\textwidth]{difficulty-graph.jpg}
  \caption{Gráfico de Dificuldade pelo tempo de meineração}
\end{figure}

\end{document}
